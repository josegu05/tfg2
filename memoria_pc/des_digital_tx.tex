\paragraph{Introducción} El objetivo del microcontrolador en la parte de transmisión, codifica mensajes según los botones pulsados.
Consiste en tres pulsadores, donde cada cual codifica un símbolo diferente, para que el receptor actúe de manera distinta según el botón pulsado. El algoritmo de comunicación entre transmisor y receptor se realiza de manera asíncrona. Para diferenciar los símbolos digitales se ha de tener en cuenta el tipo de modulación ASK, donde el 1 implica recibir señal y el 0 no se ha recibido. 
Los relojes o timers encargados de la codificación y decodificación tanto en transmisión como en recepción, deben trabajar a la misma tasa de baudios para identificar correctamente los mensajes.

\paragraph{Configuraci\'on de reloj} 
\paragraph{Codificaci\'on de los mensajes} 
La codificación de los diferentes símbolos se desarrolla de forma que, los algoritmos de codificación y decodificación se realicen de la forma más sencilla y robusta posible, teniendo en cuenta el tipo de modulación ASK.
Es por eso que cada símbolo se representa por el número de unos lógicos transmitidos de forma que si quisieramos transmitir N símbolos, la serie de codificación sería:
\begin{table}[ht]
    \centering
    \begin{tabular}{|c|c|}
        \hline
        \textbf{Symbol} & \textbf{Codification} \\ 
        \hline
        1 & 0b1 \\ 
        2 & 0b11 \\ 
        3 & 0b111 \\ 
        N & 0b$111 \ldots 1 \cdot N \text{ times} $ \\ 
        \hline
    \end{tabular}
    \caption{Codification of Digital Symbols}
\end{table}

\lstinputlisting[language=C]{/home/jose/Documents/tfg2/digital/transmisor/main.c}
