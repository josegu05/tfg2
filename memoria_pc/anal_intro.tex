\paragraph{} En este apartado se desarrolla todo lo referente a la interfaz de radiofrecuencia (RF). En primer lugar, se explica de manera analítica el funcionamiento del transmisor y receptor, para posteriormente realizar un análisis cuantitativo, realizando los cálculos correspondientes a los valores de los componentes del sistema.

\paragraph{Transmisor} El transmisor está basado en un oscilador de un solo transistor en base com\'un sintonizado por un circuito LC conocido como circuito tanque.
Los valores del condensador e inductancia del circuito tanque, poseen una frecuencia de resonancia la cual será la que se amplifique por medio de la realimentación positiva. La oscilación es cortada eléctricamente a conveniencia por medio de otro transistor, produciendo una modulación AM-ASK. El transmisor se diseña de forma que radie la mayor potencia posible, y así propagar la señal a la mayor distancia posible.

\paragraph{Receptor super-regenerativo} diseñado en 1920 se basa en el concepto de realimentación positiva. Mientras que su antecesor, el receptor regenerativo, consiste en diseñar un bucle de realimentación cuyo Al sea $A_l = 1$ referencia teoria realim, en el momento que este receptor tiene gran sensibilidad a las señales con frecuencia igual a la de diseño. El receptor regenerativo, en la práctica, es muy complicado de llevarlo a su condicion de trabajo, pues las mínimas variaciones harán que el circuito comience a oscilar o no ser tan sensible. Por este mismo hecho se desarrolla el receptor super-regenerativo, que se basa en este mismo concepto de realimentación positiva, con la diferencia que $A_l > 1$ dejandose oscilar. pasado un determinado tiempo, el circuito corta la oscilación permitiendo que el ciclo comience de nuevo. Esta señal de reinicio y paro se denomina "quench-signal". En cada inicio del periodo de la quench-signal, momento en el cual la oscilación se está montando, el circuito atraviesa un periodo de sensibilidad máxima a las señales con frecuencia igual a la de diseño. Si una señal es detectada, la oscilación del circuito se producirá de forma más rápida, aumentando así la frecuencia de la quench-signal, obteniendo como salida una señal con modulación FM con frecuencia de la quench-signal.

%\paragraph{} Eleccion de frecuencia 33MHz (sistemas de radio control) El receptor super-regenerativo trabaja mejor con frecuencias mayores pues permite una frecuencia de quench mayor, aumentando la tasa de muestreo. empiricamente cuanto mayor frecuencia mejor diseño de antena optimo mas corto, mejor distancia 
