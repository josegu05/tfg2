\paragraph{Desarrollo técnico}
Como intento de mejora a la anterior versión, la cual era de conversión directa, se trató de diseñar un receptor superheterodino. A su vez, se mantuvo el diseño del transmisor a varactores, pero se cambió la frecuencia de trabajo a una mayor, unos \SI{16}{\mega\hertz}. 
\paragraph{}
El diseño consistía en un filtro de entrada que era mezclado en un mezclador con un oscilador. Después se realizaba el tratamiento con la señal de frecuencia intermedia. Un amplificador de dos etapas y posteriormente a un rectificador con filtro paso bajo para demodular la señal. Se muestra un esquemático en la figura REF

\paragraph{Motivos de reemplazo}
buena selectividad
abandono por mala distancia de recepcion amplificador de FI complejo y oscilante, en general no era eficiente, muchos componentes para bajas prestaciones. aparte de cambio de modulacion y frecuencia. 
Es por esto, que en el diseño final se opta por diseñar un sistema AM.
