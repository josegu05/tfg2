\paragraph{Desarrollo técnico}
Como intento de mejora a la anterior versión, la cual era de conversión directa, se trató de diseñar un receptor superheterodino. A su vez, se mantuvo el diseño del transmisor a varactores, pero se cambió la frecuencia de trabajo a una mayor, unos \SI{16}{\mega\hertz}. 
\paragraph{}
El diseño consistía en un filtro de entrada que era mezclado en un mezclador con un oscilador. Después se realizaba el tratamiento con la señal de frecuencia intermedia. Un amplificador de dos etapas y posteriormente a un rectificador con filtro paso bajo para demodular la señal. Se muestra un esquemático en la figura REF

\paragraph{Motivos de reemplazo}
Este diseño funcionaba bien cuando se conectaba a la entrada un generador de frecuencias a la frecuencia de trabajo de muy baja potencia.
El problema surgía cuando se trataba de probar con el transmisor. 
El receptor no tenía buena selectividad y los amplificadores de frecuencia intermedia, los cuales no estaban correctamente diseñados, producían oscilaciones.
Gracias al mezclador, el circuito era capaz de detectar señales de muy baja potencia con muy buena selectividad. A pesar de todo, el mezclador era bastante sensible al ruido, ya que producía bastantes armónicos, producidos por efectos de segundo orden. Esto explica el por qué al conectarlo con el generador, lo más cercano a un tono puro, funcionaba correctamente, sin embargo cuando se trataba de enlazar con la señal del transmisor la cosa cambiaba.
También el diseño general, me parecía que se utilizaban demasiados componentes para unas prestaciones tan bajas. El diseño debía ser sencillo y funcional.
buena selectividad
abandono por mala distancia de recepcion amplificador de FI complejo y oscilante, en general no era eficiente, muchos componentes para bajas prestaciones. aparte de cambio de modulacion y frecuencia. 
Es por esto, que en el diseño final se opta por diseñar un sistema AM.
