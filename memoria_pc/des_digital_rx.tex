\paragraph{} El objetivo del microcontrolador en la parte de recepción tiene dos funciones. Por un lado, implementar un contador de frecuencia que identifique las variaciones recibidas por el m\'odulo de RF correspondientes a los diferentes s\'imbolos digitales, de tal forma que sirva como demodulador. Por otro lado, decodificar la señal digital recibida, identificando la orden concreta transmitida por el transmisor.

\paragraph{Descripci\'on de funcionamiento} 
\paragraph{}
El algoritmo de funcionamiento del receptor se implementa por medio de dos \textit{timers/counters} incorporados en el SOC del Atmega328p. La configuraci\'on y uso de estos dispositivos se encuentra en la hoja de datos del microcontrolador\footnote{Atmel. (n.d.). \textit{ATmega328P Automotive Microcontrollers Datasheet} (Atmel-7810). Recuperado de \url{https://www.microchip.com/wwwproducts/en/ATmega328P}.}. La estrategia de implementación es la siguiente: mientras uno de los \textit{timers} genera interrupciones periódicas en un intervalo de tiempo conocido, el segundo \textit{timer/counter} se encarga de detectar el número de flancos de subida o bajada producidos por la señal de salida modulada en FM del módulo de RF. Este proceso provoca un número de interrupciones variable en función de la frecuencia de la señal de FM en un intervalo de tiempo conocido. Este algoritmo es conocido como contador de frecuencia.

\paragraph{} Cada vez que el \textit{timer} produzca $N$ interrupciones periódicas, el c\'odigo se encargará de examinar el número de interrupciones producidas por el \textit{counter} en ese lapso de tiempo. Posteriormente, en función del número de interrupciones del counter, se decidir\'a si se ha recibido señal o no.

\paragraph{} Se hace uso tanto del TIMER0 como del TIMER2 del microcontrolador. Esto es debido a que ambos \textit{timers} poseen las mismas caracter\'isticas necesarias las cuales se encuentran expuestas en la hoja de datos. Existen, a su vez, más \textit{timers/counters} con características más complejas, pero no serán necesarias en este proyecto.
\paragraph{}
Se configura TIMER0 como temporizador, generando la interrupci\'on periódica necesaria conocida como \textit{gate}. Mediante el registro de configuración propio del \textit{timer}, se configura la frecuencia a la cual se genera esta interrupción peri\'odica. La rutina de tratamiento de interrupción ISR(TIMER0), incrementa una variable que permite conocer el n\'umero de veces que se genera la interrupci\'on.
% se encarga de comparar el número de interrupciones producidas por el counter, almacenadas en una variable global, y un número fijo umbral. Si el número de cuentas supera el umbral, la señal fue recibida, produciendo la demodulación digital. 
% PROBLEMAS ajuste tedioso muy dependiente de la frecuencia.
\paragraph{}
Por otro lado, TIMER2, se configura como contador, identificando los flancos de bajada de una señal externa introducida por el pin OSC2. La interrupción del TIMER2, se puede producir cada cierto número de flancos detectados. La rutina de tratamiento de interrupción ISR(TIMER2) incrementa la variable global de cuenta cada $M$ n\'umero de flancos detectados.

\paragraph{}
Finalmente, el objetivo de la funci\'on principal \textit{main()}, es identificar la recepci\'on de señal y actuar en consecuencia.
Para ello, el programa actúa de la siguiente forma: 
\paragraph{}
En primer lugar, espera a que la interrupci\'on del temporizador \textit{gate} se haya activado el número de veces necesaria $(N)$. Ambos contadores durante el periodo de espera se encuentran actualiz\'andose continuamente.
Después se realiza una media aritmética dividiendo el número de flancos detectados por el contador entre el número de veces que \textit{gate} se activó.
En caso de que la media calculada supere la media obtenida anteriormente, quiere decir que se ha detectado señal, por lo que el programa procede a actuar en consecuencia.
El programa implementa una máquina de estados en función del número de señales recibida.

\paragraph{Limitaciones}
\paragraph{}
Algunos problemas que surgieron a la hora de llevar a cabo este algoritmo fueron.
La velocidad de procesamiento de instrucciones debe ser decenas de veces más rápida que la frecuencia de entrada, \textit{quench-signal}. Siendo la rutina IRQ(TIMER2), del contador la función crítica.
En las primeras versiones del código recargaban mucho las rutinas de tratamiento de interrupción, escalando muy rápido este problema.
\paragraph{}
También ha sido necesario introducir una función de \textit{startup()}, ya que en el momento que se conecta el circuito a la fuente de alimentación, la frecuencia de \textit{quench} tarda un tiempo en estabilizarse. La función \textit{startup()} asegura que la señal de \textit{quench} es estable.

%\paragraph{} Lo óptimo para que la identificación de las variaciones de frecuencia fuera lo más sensible posible sería que se provocara una interrupción con cada flanco de la señal de entrada y que la interrupción periódica del timer fuera lo más extensa posible, pero nos encontramos con varios limitantes: la frecuencia de reloj de CPU y su procesamiento de instrucciones y la velocidad de transmisión de datos (bps).
%\paragraph{} Para encontrar el límite se realiza un cálculo aproximado y posteriormete se ajustan los valores del programa con ayuda de un osciloscopio. El cálculo realizado es el siguiente: CALCULO.

\paragraph{Presentaci\'on del c\'odigo}
\paragraph{}
Todas las versiones de los c\'odigos, tanto del emisor como del receptor, se encuentran en el repositorio de GitHub\footnote{\textit{https://github.com/josegu05/tfg2}}. La versi\'on final del c\'odigo del receptor se muestra a continuaci\'on. El c\'odigo se presenta en dos columnas por hoja, el orden de lectura es el siguiente: primera columna, segunda columna, hoja siguiente.

\begin{multicols}{2}
\lstinputlisting[language=C]{/home/jose/Documents/tfg2/digital/receptor-8mhz-mean/main.c}
\end{multicols}
