% gestion de ideas y su implementacion en la realidad, diversos niveles de complejidad, un proyecto complicado en la teoria en la practica no funcionaba como se esperaba, al final equilibrio entre complejidad y performance.
% Conocimiento de sistemas de radiofrecuencia y metodos de analisis y visualizacion practica, diversos errores relacionados con RF 
% Relacion de los conocimientos de las distintas ramas que ofrece el grado (control, informatica y diseno digital, fisica y electromagnetismo, radiofrecuencia, electronica analogica).
\paragraph{}
Se concluye con un análisis acerca de si se completaron los objetivos propuestos en el apartado REF.
En primer lugar, el objetivo de conocimiento de diseño de un sistema de comunicaciones lo considero satisfactorio. Al haber comprendido e interiorizado las bases de los circuitos de radiofrecuencia y puesto en práctica diversas técnicas de recepción de señales de baja potencia como mezcladores, amplificadores o filtros cada uno de ellos diseñado al nivel de componente.
Al tratarse de un diseño realizado con componentes discretos, fueron múltiples los problemas encontrados hasta lograr un diseño final. A su vez fueron múltiples también, las alternativas propuestas de diferentes desarrollos del sistema. Esto me ha permitido conocer diferentes técnicas de diseño de circuitos de radio. En general, considero satisfactorio este objetivo.
En cuanto a la puesta en práctica de los diferentes campos considero que es satisfactorio, consiguiendo relacionar los conceptos de los lazos de control en conjunto con los modelos de los componentes electrónicos y contrastados con la realidad. 
También conocimientos de la rama informática a la hora de programar los microcontroladores y establecer todo el entorno de desarrollo. Con la correspondiente conexión entre los sistemas analógicos.
Por último el diseño del transformador de la antena englobando toda la parte de radiofrecuencia, antenas y electromagnetismo. Considero satisfactorio este punto.
Finalmente, tras tantas posibles alternativas y problemas, se converge hacia un diseño sencillo y funcional con los requisitos tan exigentes que se habían propuesto del diseño desde la raíz. 

\paragraph{}
En resumen, considero este proyecto como una experiencia, aunque dura y complicada, que me ha llevado mas tiempo del esperado, ha sido satisfactoria, y que ademas me ha permitido llevar con facilidad mi desarrollo a nivel profesional, debido a la asimilacion de numerosos conceptos estudiadas en las materias y llevados a la practica.
